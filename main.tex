\documentclass{article}[12pt]


\setlength{\oddsidemargin}{0.25in}
\setlength{\textwidth}{6in}
\setlength{\topmargin}{-0.25in}

\setlength{\headheight}{0.3in}
\setlength{\headsep}{0.2in}

\setlength{\textheight}{9in}
\setlength{\footskip}{0.1in}

%%%%%%%%%%%%%%%%%%%%%%%%%%%%%%%%%%%%%%%%%%%%%%%%%%%%%%%%%%%%%%%%%%%%%%%%%%%%%%%%%%%%%%%%%%%%%%%%
\usepackage[latin1]{inputenc}
\usepackage{verbatim}
\usepackage{amsmath}
\usepackage{amssymb}
\usepackage{amsthm}
\usepackage{comment}
\usepackage{setspace}
\usepackage{tikz}
\usepackage{tikz-qtree}
\usetikzlibrary{shapes,decorations,trees}



% Define box and box title style
\tikzstyle{mybox} = [draw=red, fill=blue!20, very thick,
    rectangle, rounded corners, inner sep=10pt, inner ysep=20pt]
\tikzstyle{fancytitle} =[fill=red, text=white]
\tikzstyle{defnbox} = [draw=blue, fill=red!20, very thick,
    rectangle, rounded corners, inner sep=10pt, inner ysep=20pt]
\tikzstyle{fancydefn} =[fill=blue, text=white]

% Define (boxed) theorem
\newenvironment{thm}[2]{
\begin{tikzpicture}
\node [mybox] (box){%
    \begin{minipage}{\textwidth}
#2
    \end{minipage}
};
\node[fancytitle, rounded corners, right=12pt] at (box.north west) {#1};
\end{tikzpicture}%
}

% Define (boxed) definition
\newenvironment{defn}[2]{
\begin{tikzpicture}
\node [defnbox] (box){%
    \begin{minipage}{\textwidth}
#2
    \end{minipage}
};
\node[fancydefn, rounded corners, right=12pt] at (box.north west) {#1};
\end{tikzpicture}%
}







\theoremstyle{plain}
\newtheorem{theorem}{Theorem}
\newtheorem*{theorem*}{Theorem}
\newtheorem{corollary}[theorem]{Corollary}
\newtheorem{lemma}[theorem]{Lemma}
\newtheorem{proposition}[theorem]{Proposition}

\theoremstyle{definition}
\newtheorem*{example}{Example}
\newtheorem*{definition}{Definition}
\newtheorem*{remark}{Remark}
\newtheorem*{caveat}{Caveat}
\newtheorem*{motivation}{Motivation}

\newcommand{\al}{\alpha}
\newcommand{\be}{\beta}
\newcommand{\De}{\Delta}

\newcommand{\K}{\mathop{\raisebox{-5pt}{\huge \textnormal{K}}}}

\newcommand{\fold}{\mathop{\raisebox{-0.2ex}{\huge $\between$}}}

\newcommand{\cfplus}{\raisebox{-6pt}{$+$}}
\newcommand{\cfminus}{\raisebox{-6pt}{$-$}}

\newcommand{\N}{\mathbb{N}}
\newcommand{\Z}{\mathbb{Z}}
\newcommand{\R}{\mathbb{R}}
\newcommand{\Q}{\mathbb{Q}}
\newcommand{\C}{\mathbb{C}}
\newcommand{\UH}{\mathbb{H}}
\newcommand{\Cc}{\hat{\mathbb{C}}}

\newcommand{\M}{\mathcal{M}}

\DeclareMathOperator{\Tr}{Tr}

\newcommand{\lp}{\left(}
\newcommand{\rp}{\right)}

%%%%%%%%%%%%%%%%%%%%%%%%%%%%%%%%%%%%%%%%%%%%%%%%%%%%%%%%%%%%%%%%%%%%%%%%%%%%%%%%%%%%%%%%%%%%%%%%
%\onehalfspacing
\begin{document}

\makeatletter	   % `@' is now a normal "letter' for LaTeX
\renewcommand{\ps@plain}{%
     \renewcommand{\@oddhead}{\textrm{Arnold}\hfil\textrm{\thepage}}% 
     \renewcommand{\@evenhead}{\@oddhead}%
     \renewcommand{\@oddfoot}{}% empty footer
     \renewcommand{\@evenfoot}{\@oddfoot}}
\makeatother     % `@' is restored as a "non-letter" character


%%%%%%%%%%%%%%%%%%%%%%%%%%%%%%%%%%%%%%%%%%%%%%%%%%%%%%%%%%%%%%%%%%%%%%%%%%%%%%%%%%%%%%%%%%%%%%%%
\title{Example of a Purely Periodic \\ Quatrenary Continued Fraction Over $\Z[i]$ }        
\author{Bryan Arnold}               
\maketitle

\pagestyle{plain}
%%%%%%%%%%%%%%%%%%%%%%%%%%%%%%%%%%%%%%%%%%%%%%%%%%%%%%%%%%%%%%%%%%%%%%%%%%%%%%%%%%%%%%%%%%%%%%%%
Let $\al = \sqrt[4]{2+i}$ be the principal root. And let $\tau = 1+\al+\al^2+\al^3$ (nearly $5.38912+1.10811 i$). Then
$$
\begin{tikzpicture}
\tikzset{grow'=right,level distance=60pt}
\tikzset{execute at begin node=\strut}
\tikzset{every tree node/.style={anchor=base west}}
\Tree 
[.$\tau=4$ [.$6+6i$ [.$8i$ [.$-2+2i$ ] [.$4$  [.$\dots$ ] [.$\dots$ ] ] ] [.$4$ [.$6+6i$ [.$\dots$ ] [.$\dots$ ] ] [.$4$  [.$\dots$ ] [.$\dots$ ] ] ] ]
[.$4$ [.$6+6i$ [.$8i$ [.$-2+2i$ ] [.$\dots$ ] ] [.$4$  [.$\dots$ ] [.$\dots$ ] ] ]
[.$4$ [.$6+6i$ [.$\dots$ ] [.$\dots$ ] ]
[.$4$ [.$\dots$ ] [.$\dots$ ] ] ] ] ]
\end{tikzpicture}
$$
where we denote
$$
\begin{tikzpicture}
\tikzset{grow'=right,level distance=55pt}
\tikzset{execute at begin node=\strut}
\tikzset{every tree node/.style={anchor=base west}}
\Tree 
[.$p+\cfrac{q}{r}=p$ [.$q$ ] [.$r$ ] ]
\end{tikzpicture}
.$$
More precisely, we have $A_n/D_n \to \tau$ as $n \to \infty$, where
$$
\begin{pmatrix}
A_n \\
B_n \\
C_n \\
D_n  
\end{pmatrix}
=
\begin{pmatrix}
4 & 1 & 0 & 0 \\
6+6i & 0 & 1 & 0 \\
8i & 0 & 0 & 1 \\
1 & 0 & 0 & 0 
\end{pmatrix}
\begin{pmatrix}
4 & 1 & 0 & 0 \\
6+6i & 0 & 1 & 0 \\
8i & 0 & 0 & 1 \\
-2+2i & 0 & 0 & 0 
\end{pmatrix}^n
\begin{pmatrix}
1 \\
0 \\
0 \\
0 
\end{pmatrix}.
$$

\end{document}